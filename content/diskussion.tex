\section{Diskussion}
\label{sec:Diskussion}
Die Bohrung des Acrylblocks lassen sich per Ultraschall in guter Näherung bestimmen.
Es muss beachtet werden, dass Abweichungen durch die Schutzschicht auf den Sonden entstehen können.
Zusätzlich konnte die Bohrung $10$, von der Unterkante zur Oberkante gemessen, nicht loklisiert werden.
Dies ist auf die Lage der Bohrung zurückzuführen.
Die Bohrung $11$ überdeckt Bohrung $10$ vollständig, da sie einen größeren Durchmesser aufweist.


\noindent Aus den erhaltenen Werten für die Bestimmung des Auflösungsvermögens lässt sich folgern,
dass eine Messung mit höherer Frequenz eine genauere lokalisierung der Bohrungen ermöglicht.
Die Entfernung für welche die Messung eingesetzt wird nimmt exponentiell ab.
Daher konnten nur Werte für die Messung von Oberkante zur Unterkante gemessen werden.


\noindent Die Messung mit dem sehr einfachen Herzmodell hat gezeigt, dass sich mittels eines TM-Scans der Herzvolumenstrom mit geringer Unsicherheit 
bestimmt werden kann, sodass sich dieses Verfahren besonders für die Untersuchung der Funktionsfähigkeit eines echten Herzens eignet.
Dies trifft insbesondere zu, da es sich um ein nichtinvasives Verfahren handelt und lebende Organismen nicht beschädigt werden.
