\section{Diskussion}
\label{sec:Diskussion}
Die Lage Bohrung des Acrylblocks lassen sich per Ultraschall in guter Näherung bestimmen.
Diese weichen von $0.6\%$ bis ungefähr $5\%$ ab.
Dabei ist zu beachten, dass die Messwerte des B-Scans zu größeren Abweichungen tendieren als die des A-Scans.
Die Durchmesser lassen sich nur mühselig per Ultraschall bestimmen.
Aufgrund der geringen Durchmesser fallen schon kleine Abweichungen der Strecken stark auf.
So kann es passieren, dass einige Durchmesser mit den Werten der Schieblehre übereinstimmen und andere um über $85\%$ abweichen.
Die negativen Werte für zwei der Durchmesser dürfen nicht als solche interpretiert werden, sondern sind resultat der Messungenauigkeit.
Es muss beachtet werden, dass Abweichungen durch die Schutzschicht auf den Sonden entstehen können.
Zusätzlich konnte die Bohrung $10$ von der Unterkante gemessen nicht loklisiert werden.
Dies ist auf die Lage der Bohrung zurückzuführen.
Die Bohrung $11$ überdeckt Bohrung $10$ vollständig, da sie einen größeren Durchmesser aufweist.
Eine Erhöhung des Auflösungsvermögens, durch eine größere Ultraschall-Frequenz, konnte nicht erziehlt werden.
Die gemessenen Werte von der Oberkante sind identisch mit den Werten der $2\si{\mega\hertz}$ Sonde.
Es konnten keine Werte von der Unterkante gemessen werden.
Das lässt sich darauf zurückführen, dass die Eindringungstiefe des Ultraschalls mit zunehmender Frequenz abnimmt.
\noindent Die Messung mit dem sehr einfachen Herzmodell hat gezeigt, dass sich mittels eines TM-Scans der Herzvolumenstrom mit geringer Unsicherheit
bestimmt werden kann, sodass sich dieses Verfahren besonders für die Untersuchung der Funktionsfähigkeit eines echten Herzens eignet.
Dies trifft insbesondere zu, da es sich um ein nichtinvasives Verfahren handelt und lebende Organismen nicht beschädigt werden.
