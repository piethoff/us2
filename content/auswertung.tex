\section{Auswertung}
\label{sec:Auswertung}
\subsection{Untersuchung des Acrylblocks mit einer Schieblehre}
Die gemessene Höhe des Acrylblocks beträgt:
\begin{equation}
	\label{block_schieb}
  h =  \SI{80.25}{\milli\meter}
\end{equation}
Die Bohrungen des Acrylblocks werden mit einer Schieblehre vermessen.
Die erhaltenen Werte sind in Tabelle \ref{tab:schieb} eingetragen.
\begin{table}[H]
    \caption{Messung der Borungen mit einer Schieblehre.}
    \label{tab:schieb}
    \centering
    \begin{tabular}{S[table-format=2] S[table-format=2.2(0)e0] S[table-format=2.2(0)e0] S[table-format=2.2(0)e0]  }
        \toprule
        {Bohrung} & {Oberkante$/\si{\milli\meter}$} & {Unterkante$/\si{\milli\meter}$} &{Durchmesser$/\si{\milli\meter}$} \\
        \midrule
             1 & 19.20  & 59.65 & 1.40\\
             2 & 17.45  & 61.35 & 1.45\\
             3 & 61.05  & 13.45 & 5.75\\
             4 & 53.70  & 21.70 & 4.85\\
             5 & 46.30 & 30.95 & 3.00\\
             6 & 38.70 & 38.65 & 2.90\\
             7 & 30.75 & 46.65 & 2.85\\
             8 & 22.80 & 54.65 & 2.80\\
             9 & 14.80 & 62.70 & 2.75\\
             10 & 6.85 & 70.65 & 2.75\\
             11 & 56.40 & 15.20 & 8.65\\
        \bottomrule
    \end{tabular}
\end{table}
\noindent
\subsection{Untersuchung des Acrylblocks mit A-Scan}
Der Acrylblock wird wie in der Durchführung beschrieben mit dem A-Scan vermessen.
Die Schallgeschwindigkeit in Acryl beträgt $c=\SI{2730}{\meter\per\second}$\cite{acryl}.
Die Eindringungstiefe kann, dann nach Gleichung \eqref{eq:schall} berechnet werden.
Die erhaltenen Werte sind in Tabelle \ref{tab:a-scan} eingetragen.
Dabei muss die Abweichung, welche durch die Schutzschicht der Sonde entsteht, beachtet werden.
Für den gesamten Blockdurchmesser ergibt sich ein Wert von
\begin{equation}
	\label{block_scan}
  h_\text{Scan} = \SI{80.50}{\milli\meter}
\end{equation}
Aus der Differenz der Messung mit der Schieblehre \eqref{block_schieb} und der Messung mit dem Scan \eqref{block_scan} ergibt sich die Schichtdicke:
\begin{equation*}
  \Delta h = \SI{0.25}{\milli\meter}
\end{equation*}
Diese wird im Folgenden von allen Messwerten abgezogen, um die tatsächliche Eindringungstiefe zu erhalten.
\begin{table}[H]
    \caption{Messung der Bohrungen mit dem A-Scan .}
    \label{tab:a-scan}
    \centering
    \begin{tabular}{S[table-format=2] S[table-format=2.2(0)e0] S[table-format=2.2(0)e0] S[table-format=2.2(0)e0] }
        \toprule
        {Bohrung} & {Oberkante$/\si{\milli\meter}$} & {Unterkante$/\si{\milli\meter}$}&{Durchmesser$/\si{\milli\meter}$} \\
        \midrule
             1 & 19.82  & 60.58 & -0.15\\
             2 & 17.70  & 61.13 & 1.42\\
             3 & 59.95  & 13.22 & 7.08\\
             4 & 53.68  & 21.61 & 4.96\\
             5 & 46.27 & 29.99 & 3.99 \\
             6 & 38.75 & 38.65 & 2.85\\
             7 & 30.64 & 46.67  & 2.94\\
             8 & 22.77 & 54.52 & 2.96 \\
             9 & 14.77 & 62.73 &  2.75 \\
             10 & 7.01 & / & /\\
             11 & 55.43 &  15.08  & 9.74\\
        \bottomrule
    \end{tabular}
\end{table}
\noindent
Erhält ein Element der Tabelle ein "/", so konnte kein Wert an der entsprechenden Stelle gemessen werden.
Die Bohrungen $1$ und $2$ werden erneut vermessen. Die dafür verwendete Sonde besitzt ein Auflösungsvermögen von $\SI{4}{\mega\hertz}$.
Die Werte sind in Tabelle \ref{tab:aufl} eingetragen.
\begin{table}[H]
    \caption{Messung des Auflösungsvermögen.}
    \label{tab:aufl}
    \centering
    \begin{tabular}{S[table-format=2] S[table-format=2.2(0)e0] S[table-format=2.2(0)e0]  }
        \toprule
        {Bohrung} & {Oberkante$/\si{\milli\meter}$} & {Unterkante$/\si{\milli\meter}$} \\
        \midrule
             1 & 19.82  & /\\
             2 & 17.70  & /\\
        \bottomrule
    \end{tabular}
\end{table}
\noindent
\subsection{Untersuchung des Acrylblocks mit B-Scan}
Die Messung wird wie in der Durchführung beschrieben ausgeführt.
\begin{figure}[H]
  \centering
  \includegraphics[width=0.8\textwidth]{content/bscan-2mhz_oberseite.png}
  \caption{Messung des Acrylblock von der Oberkante.}
  \label{fig:obs}
\end{figure}
\begin{figure}[H]
  \centering
  \includegraphics[width=0.8\textwidth]{content/bscan-2mhz_unterseite.png}
  \caption{Messung des Acrylblock von der Unterkante.}
  \label{fig:unts}
\end{figure}
\noindent Aus der Aufnahme \ref{fig:obs} und der Aufnahme \ref{fig:unts} können die Bohrungen lokalisiert werden.
Die erhaltenen Werte sind in Tabelle \ref{tab:b-scan} eingetragen.
\begin{table}[H]
    \caption{Messung der Bohrungen mit dem B-Scan .}
    \label{tab:b-scan}
    \centering
    \begin{tabular}{S[table-format=2] S[table-format=2.2(0)e0] S[table-format=2.2(0)e0] S[table-format=2.2(0)e0]  }
        \toprule
        {Bohrung} & {Oberkante$/\si{\milli\meter}$} & {Unterkante$/\si{\milli\meter}$}&{Durchmesser$/\si{\milli\meter}$} \\
        \midrule
             1 & 20.18  & 59.92 & 0.15\\
             2 & 18.58  & 61.85 & -0.18\\
             3 & 61.68  & 13.64 &  4.93\\
             4 & 54.02  & 22.57 & 3.66\\
             5 & 46.68 & 30.71  & 2.86 \\
             6 & 39.16 & 39.34  & 1.75\\
             7 & 31.35 & 47.15  & 1.75\\
             8 & 23.37 & 55.13  & 1.75\\
             9 & 15.71 & 62.27  & 2.27\\
             10 & 8.05 & / & /\\
             11 & 56.01 &  15.87 &  8.37\\
        \bottomrule
    \end{tabular}
\end{table}
\noindent
\subsection{Untersuchung des Herzmodells}
Für das Herz wird einen Radius von $r_\text{Herz}=\SI{22.75}{\milli\meter}$ gemessen.
Als Schallgeschwindigkeit für Wasser wurde ein Wert von $c=\SI{1484}{\meter\per\second}$\cite{wasser} verwendet.
Die Werte für das Grundniveau und den Maximalausschlag sind in Tabelle \ref{maxschlag} aufgetragen.

\begin{table}[H]
    \caption{Messung des maximalen Ausschlags.}
    \label{maxschlag}
    \centering
    \begin{tabular}{S[table-format=2.2] S[table-format=2.2]}
        \toprule
        {Grundniveau$/\si{\milli\meter}$} &
        {Maximaler Ausschlag$/\si{\milli\meter}$} \\
        \midrule
        38.27             & 6.38 \\
        \bottomrule
    \end{tabular}
\end{table}
\noindent
Die Oberfläche der Kammer des Herzmodells wird als Kugelschale genähert.
Das Volumen eines Kugelschnittes lässt sich wie folgt berechnen:
\begin{align}
	\label{kugelvol}
    V &= \frac{\pi h^2}{3}\left( \frac{3({r'}^2+h^2)}{2h} - h \right) \\
    \symup{\Delta}V &= \frac{\pi h^2}{2}(3{r'}^2+5h^2-2)\symup{\Delta}h
\end{align}
Wobei $r'$ der Radius der Herzmodells und $h$ die Auslenkungshöhe ist.
Für die Messung mittels des A-Scans ergibt sich somit ein Herzvolumen von:
\begin{equation}
	V_\text{A-Scan} = \SI{4.29}{\num{-e5}\meter\cubed}
\end{equation}
Anschließend wird der simulierte Herzschlag untersucht und die Messerbegnisse in Tabelle \ref{tab:schlag} aufgetragen.
Von den Messwerten ist bereits das Grundniveau abgezogen.
\begin{table}[H]
    \caption{Messung des Herzschlags.}
    \label{tab:schlag}
    \centering
    \begin{tabular}{S[table-format=2.2]}
        \toprule
        {Peaks ohne Grundniveau$/\si{\milli\meter}$} \\
        \midrule
            14.87 \\
            13.96 \\
            15.98 \\
            14.26 \\
            13.46 \\
            14.57 \\
            14.17 \\
            15.18 \\
            14.07 \\
            14.37 \\
            14.97 \\
            14.68 \\
            13.56 \\
            14.57 \\
            13.46 \\
            14.47 \\
            14.87 \\
            14.87 \\
            16.49 \\
            12.85 \\
        \bottomrule
    \end{tabular}
\end{table}
\noindent
Der Herzschlag wurde über eine Zeit von $\SI{27.8}{\second}$ gemessen.
Der Mittelwert wird gebildet nach der Vorschrift:
\begin{equation}
	\overline{h} = \frac{\sum_i^\text{Anzahl der Messwerte}h_i}{\text{Anzahl der Messwerte}}
\end{equation}
Es wird zusätzlich eine Hilfsgröße eingeführt:
\begin{equation}
	\delta h_i = h_i-\overline{h}
\end{equation}
Die Standardabweichung kann geschrieben werden als:
\begin{equation}
	\symup\Delta\overline{h} = \sqrt{\frac{\sum_i^\text{Anzahl der Messwerte}(\delta h)^2}{(\text{Anzahl der Messwerte})(\text{Anzahl der Messwerte} - 1)}}
\end{equation}
Es ergibt sich somit eine mittlere Schlagauslenkung von
\begin{equation}
   \overline{h}_\text{Herz} = \SI{14.48\pm0.19}{\milli\meter}
\end{equation}
und
\begin{equation}
   \overline{\nu}_\text{Herz} = \SI{0.72}{\hertz}.
\end{equation}
Es ergibt sich nach Gleichung \eqref{kugelvol} ein mittlerer Herzstrom von
\begin{equation}
   \overline{Q}_\text{Herz} = V\left(\overline{h}_\text{Herz}\right)\overline{\nu}_\text{Herz} = \SI{9.62\pm0.09e-6}{\meter\cubed\per\second}.
\end{equation}
