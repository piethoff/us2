\section{Zielsetzung}
Ziel dieses Versuchs ist es, Materialien mit Hilfe von Ultraschall zu untersuchen.
Dabei sollen diese ein, zwei und drei dimensional betrachtet werden.
\section{Theorie}
\label{sec:Theorie}
Ultraschall befindet sich in dem Spektrum von 20 kHz bis 1 GHz.
Schall mit größeren Frequenzen wird als Hyperschall bezeichnet und Schall mit geringeren Frequenzen als Infraschall.
Schall bewegt sich als longitudinale Welle gemäß
\begin{equation}
  p(x,t) = p_0 + \nu_0 Z \cos{(\omega t - k x)}
\end{equation}
 aufgrund von Durckschwankungen fort.
 $Z=c \rho$ beschreibt hier die akustische Impedanz, welche durch Dichte und Schallgeschwindigkeit in dem Material bestimmt wird.
In Flüssigkeiten lässt sich die Schallgeschwindigkeit aus der Kompressibilität $\kappa$ und der Dichte $\rho$ wie folgt bestimmen:
 \begin{equation}
  c_\text{Fl}=\sqrt{\frac{1}{\kappa \rho}}   
 \end{equation}
 In Festkörpern berechnet sich die Schallgeschwindigkeit, anders als in Flüssigkeiten, nach:
 \begin{equation}
  c_\text{Fe}=\sqrt{\frac{E}{\rho}}   
 \end{equation}
Wobei $E$ das Elastizitätsmodul des entsprechenden Körpers ist.
Die Schallausbreitung in einem Festkörper gestaltet sich schwieriger als in einer Flüssigkeit oder einem Gas.
Es sind zusätzlich transversale Wellen möglich.
Es resultiert, dass die Schallgeschwindigkeit im Festkörper richtungsabhängig ist und die Intensität der Welle mit der Strecke $x$ exponentiell abnimmt:
\begin{equation}
  I(x) = I_0 \exp{- \alpha x}
\end{equation}
dargestellt. Dabei stellt $\alpha$ den Absobtionskoeffizienten dar.
\\Für die Untersuchung wird in der Regel ein Kontakmittel verwendet, da der Ultraschall von der Luft stark absorbiert wird.
Der Reflexionskoeffizient berechnet sich dann mit:
\begin{equation}
  R=\left(\frac{Z_1 - Z_2}{Z_1 + Z_2}\right)^2
\end{equation}
Dieser beschreibt das Verhältnis von einfallender zu reflektierter Schallintensität und wird mit der Impedanz der angrenzenden Materialien bestimmt.
Der transmittierte Anteil lässt sich mit
\begin{equation}
  T= 1-R
\end{equation}
bestimmen.
\subsection{Erzeugung von Ultraschall und Analyse-Verfahren}
Ultraschall wird häufig mit piezoelektrischen Kristallen erzeugt.
Wird ein solcher Kristall in ein elektrisches Wechselfeld gebracht, beginnt dieser zu schwingen.
Der Kristall kann auch als Empfänger verwendet werden. Dabei treffen die Schallwellen auf den Kristall und regen ihn so zum Schwingen an.
Im Allgemeinen wird zwischen zwei Verfahren in der Ultraschalltechnik verwendet, nämlich dem Durchschallungs-Verfahren und dem Impuls-Echo-Verfahren.
Für das Durchschallungs-Verfahren wird ein Sender an dem einen Ende und ein Empfänger an dem anderen Ende der zu untersuchenden Probe platziert.
Befindet sich eine Fehlstelle in dieser, wird die Intensität abgeschwächt.
Der genaue Ort der Fehlstellen kann nicht festgestellt werden.
Beim Impuls-Echo-Verfahren wird der Sender zusätzlich als Empfänger verwendet.
Bei Fehlstellen kann die Höhe des Echos Aufschluss über die Größe der Fehlstelle geben.
Die Lage der Fehlstelle kann bei bekannter Schallgeschwindigkeit mit
\begin{equation}
  \label{eq:schall}
  s= \frac{1}{2}ct
\end{equation}
bestimmt werden.
\\Die Laufzeitdiagramme können in A-Scan, B-Scan und TM-Scan dargestellt werden.
Der A-Scan oder Amplitude-Scan ist ein eindimensionales Verfahren.
Es werden die Echo-Amplituden zusammen mit der Zeit in ein Diagramm aufgetragen.
Der B-Scan oder Brightness-Scan ist ein zweidimensionales Verfahren.
Die Echo-Amplituden werden zusammen mit die Helligkeit aufgetragen, um ein zweidimensionales Schnittbild erzeugt.
Der TM-Scan oder Time-Motion-Scan kann durch schnelle Abtastung eine Bildfolge aufnehmen, um beispielsweise Organbewegungen sichtbar zu machen.
