\section{Zielsetzung}
Ziel dieses Versuchs ist Materialien mit Hilfe von Ultraschall zu Untersuchen.
Dabei sollen diese ein, zwei und drei Dimensional betrachtet werden.
\section{Theorie}
\label{sec:Theorie}
Ultraschall befindet sich in dem Spektrum von 20 kHz bis 1 GHz.
Schall mit größeren Frequenzen wird Gigaschall bezeichnet und Schall mit geringeren Frequenzen Intraschall.
Schall bewegt sich als longitudinale Welle gemäß
\begin{equation}
  p(x,t) = p_0 + \nu_0 Z \cos{(\omega t - k x)}
\end{equation}
 aufgrund von Durckschwankungen fort.
 $Z=c \rho$ beschreibt hier die akustische Impedanz, welche durche Dichte und Schallgeschwindigkeit in dem Material bestimmt wird.
 In Festkörpern berechnet sich die Schallgeschwindigkeit, anders als in Flüssigkeiten, nach:
 \begin{equation}
  c=\sqrt{\frac{E}{\rho}}   .
 \end{equation}
Wobei $E$ das Elastizitätsmodul des entsprechenden Körpers ist.
Die Schallausbreitung in einem Festkörper gestaltet sich zusätzlich schwieriger als in einer Flüssigkeit oder einem Gas.
Es sind zusätzlich transversal Wellen möglich.
Es resultiert, dass die Schallgeschwindigkeit im Festkörper Richtungsabhängig ist und die Intensität der Welle mit der Strecke $x$ abnimmt.
Die exponentielle Abnahme ist in
\begin{equation}
  I(x) = I_0 \exp{\alpha x}
\end{equation}
dargestellt.
Für die Untersuchung wird in der Regel ein Kontakmittel verwendet, da der Ultraschall von der Luft gut absorbiert wird.
Der Reflexionskoeffizient berechnet sich dann mit:
\begin{equation}
  R=\left(\frac{Z_1 - Z_2}{Z_1 + Z_2}\right)^2
\end{equation}
Dieser beschreibt das Verhältniss von einfallender zu reflektierter Schallintensität und wird mit der Impedanz der angrenzenden Materialien bestimmt.
\subsection{Erzeugung von Ultraschall und Analyse-Verfahren}
Ultraschall wird häufig mit piezoelektrischen Kristallen erzeugt.
Wird ein solcher Kristall in ein elektrisches Wechselfeld gebracht beginnt dieser zu Schwingen.
Dieser kann auch als Empfänger verwendet werden. dabei treffen die Schallwellen auf den Kristall und regen ihn so zum Schingen an.
Im allgemeinen wird zwischen zwei Verfahren in der Ultraschalltechnik verwendet, nämlich das Durchschallungs-Verfahren und das Impuls-Echo-Verfahren.
Für das Durchschallungs-Verfahren wird ein Sender an dem einen Ende und ein Empfänger an dem anderen Ende der zu untersuchenden Probe platziert.
Befindet sich eine Fehlstelle in diesem, wird die Intensität abgeschwächt.
Der genaue Ort dieser kann nicht vestgestellt werden.
Beim Impuls-Echo-Verfahren wird der Sender zusätlich als Empfänger verwendet.
Bei Fehlstellen kann die Höhe des Echos Aufschluss über die Größe der Fehlstelle geben.
Die Lage der Fehlstelle kann, dann bei bekannter Schallgeschwindigkeit mit
\begin{equation}
  s= \frac{1}{2}ct
\end{equation}
bestimmt werden.
Die Laufzeitdiagramme können in A-Scan, B-Scan und TM-Scan dargestellt werden.
Der A-Scan oder Amplitude-Scan ist ein eindimensionales Verfahren.
Es werden die Echo-Amplituden gegen die Zeit aufgetragen.
Der B-Scan oder Brightness-Scan ist ein zweidimensionales Verfahren.
Die Echo-Amplituden werden gegen die Helligkeit aufgetragen, um ein zweidimensionales Schnittbild erzeugt.
Der TM-Scan oder Time-Motion-Scan kann durch schnelle Abtastung eine Bildfolge aufnehmen, um beispielsweise Organbewegungen sichtbar zu machen.
